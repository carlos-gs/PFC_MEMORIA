\cleardoublepage
\appendix
\chapter{Instalación y Uso}
\label{app:instalacion}


Con el sistema Misture propuesto, si se desea efectuar una corrección con una parte o la totalidad de las funcionalidades básicas ya implementadas, sería necesario:

\begin{enumerate}
\item El directorio del profesor donde deposita los ficheros auxiliares a la corrección, tales como:

\begin{itemize}
\item Fichero con el listado de logines únicos de alumnos, emails, y ruta o url del repositorio individual. En este caso obligatorio.

\item Fichero con las preguntas estáticas de la prueba o examen que se mezclaran con las generadas de autoría.

\item Ficheros auxiliares para la ejecución de pruebas sobre el código: códigos de un servidor o cliente de prueba para lanzarlo contra el del alumno, xmls de configuración, ficheros de entrada sobre el que lanzar una prueba, etc.
\end{itemize}


\item Una copia del \textit{script} principal, que lee las configuraciones e invoca los diferentes submódulos de corrección.

\begin{itemize}
\item Se utiliza a modo de plantilla, teniendo que ajustar la configuración y código del \textit{script}.

\item Estos ajustes a los módulos que realmente se van a utilizar entre aquellos que son opcionales, que llamadas se van a ejecutar en las pruebas y que resultado u listado de resultados válidos se pueden esperar.

\item Asimismo, se pueden ajustar otras constantes como el listado de errores o tipo de errores de Pep8, Pylint o Hackings que se desean ignorar en un análisis de estilo y código.

\item También, en caso de activar la comprobación de XML o PDML, se indicarían que comprobaciones, entre las implementadas, se deberían realizar, como la cuenta de trazas de determinado tipo en la captura Wireshark, la presencia de cierto atributo o valor en una etiqueta concreta, etc.
\end{itemize}
\end{enumerate}

\cleardoublepage

\chapter{Requisitos}
\label{app:req}

A continuación se especifican las dependencias necesarias para la ejecución del código de Misture.

En cuanto a librerías Python, si no se indica lo contrario, se aconseja realizar con gestor de paquetes de Python llamado Pip para instalar los paquetes necesarios en la versión 3 de Python, de la siguiente manera.
\begin{center}
\begin{verbatim}
$ pip3 install <nombre_paquete>
\end{verbatim}
\end{center}


Dada la cantidad de paquetes y dependencias manejados, se detallarán las dependencias principales, siendo instaladas todas las dependencias en las versiones disponibles o las requeridas por el propio paquete que las tiene en su lista de dependencias requeridas.


Además, por la misma razón, es una buena práctica configurar un \textit{virtual environment} de Python específico para la configuración de todo el entorno necesario y la ejecución de este proyecto.
\begin{center}
\begin{verbatim}
 $ sudo pip install virtualenv
 $ cd <directorio_ubicar_env>
 $ virtualenv ENV
 $ source ENV/bin/activate
\end{verbatim}
\end{center}

Veremos que aparece (ENV) al principio de nuestro prompt. En ese entorno se pueden instalar en Python todas las librerías necesarias, y lanzar el proyecto. Para abandonar este entorno y volver al habitual, invocamos, la siguiente instrucción, desapareciendo la cadena (ENV) de nuestro prompt.
\begin{center}
\begin{verbatim}
$ deactivate
\end{verbatim}
\end{center}

Una vez aclarado todo esto, definimos los principales requisitos:

\begin{itemize}
\item Interprete de Python:	Durante el desarrollo, se ha empleado principalmente el interprete de la versión 3.4, aconsejándose la instalación de la versión 3.4 o superior.

\item Asimismo, se dispuso de un interprete de la versión 2, en concreto de la versión 2.7, para el testing sobre códigos de Python 2. Aconsejándose, en caso de la corrección de prácticas en esa versión de Python, de un entorno Python con la versión más reciente utilizada por los alumnos.

\item Sistema gestor de bases de datos MongoDB: Se ha empleado la versión 3.2.13.

\item La librería Pymongo y el ODM PyMODM de Python: PyMongo 3.4, siendo válida cualquier versión con numeración mayor o igual a la de MongoDB empleada.
	Usar la versión 0.4.0 o superior de PyMODM, teniendo en cuenta en la documentación de esta librería, que se indica que esta probado para la versión de MongoDB que se tiene instalada.	

\item Sistema de control de versiones Git: Se recomienda tener instalada la versión 2 de este herramienta, ya que esta documentado y asimismo hemos detectado en diferentes pruebas que algunos de los comandos y opciones de Git han modificado su comportamiento, sus opciones por defecto, o el formato de su salida, afectando por tanto al comportamiento de Pygit2.

\item Librería Pygit2. El sistema Misture esta probado con la versión 0.25.0

\item Librería Beautiful Soup.
	Se debe instalar Beautiful Soup 4.
	Automáticamente se instalan el resto de dependencias.

\item Librería Python de la herramienta Flake8
	Se ha empleado la versión reciente 3.3.0.
	Automáticamente se instalan el resto de dependencias.
	
\item Librería Hacking (Extensión para Flake8): Empleada la versión 0.13

\item Librería Pep8: Empleada la versión 1.7

\item Librería Pylint: Empleada la versión 1.7.1

\item Librería Validators: Se ha empleado la versión 0.11.3

\item Librería Astor: Versión >= 0.5
\end{itemize}
