\chapter{Introducción}
%\addcontentsline{toc}{chapter}{Summary} % si queremos que aparezca en el índice
\label{sec:intro}

En este capítulo inicial vamos a proceder con la presentación del Proyecto de fin de Carrera que hemos denominado “MiStuRe: MINADO DE REPOSITORIOS DE ESTUDIANTES ORIENTADO AL ANÁLISIS DEL APRENDIZAJE”.\\


Con posteridad a la presentación desentrañaremos la motivación personal con la que se empezó este proyecto, y describiremos la estructura de este documento, de modo que el lector tenga un mapa general de aquello que se va a encontrar a continuación, y adecue su lectura según su interés.



\section{Presentación}

MiStuRe es, en primer lugar, el acrónimo de MIning STUdent REpositories, que guarda semejanza gráfica o sonora con los términos español mistura e inglés mixture.


Misture representa un proyecto de módulos desarrollados en el lenguaje Python en su versión 3, que intenta sintetizar una estructura que proporcione un modelo de datos, una interfaz o interfaces, y unos módulos fácilmente extensibles que permitan la automatización de la corrección de prácticas de programación y que recojan con la mayor riqueza posible la información para facilitar que el alumno tenga reportes frecuentes e instruyentes acerca de su desempeño y progresión, además de poder utilizarse sistemáticamente para reportes o análisis posteriores.


La utilidad del proyecto se engloba dentro de un ámbito educativo universitario en el que se desea poder realizar por medio de \textit{scripting} tareas de forma rápida y automática, y facilitar el análisis que permita detectar eficazmente los errores de los alumnos, e incluso de permitir incidir en otras destrezas del alumno, como la habilidad programando en cierto lenguaje, con cierto orden, simplicidad o calidad.


En este análisis se considera la ejecución de pruebas sobre el código o de comprobaciones sobre los ficheros entregables, para evaluar que cumplen con la funcionalidad pedida o que demuestren que la tarea ha sido efectuada correctamente, lo que implica en general un buen aprovechamiento de la actividad por parte del alumno.


Asimismo, se consideran otros análisis o comprobaciones, tales como el análisis de trazas de Wireshark, útil para la corrección de actividades relacionadas con el uso o análisis de redes, protocolos o servicios web o multimedia.


Otro de los análisis de interés es el del uso de la herramienta Git, que en primera instancia se puede aprovechar para conducir al alumno a la adquisición de ciertos hábitos a la hora de manejar estas herramientas con su código fuente. Y en instancias superiores, a un posible análisis pormenorizado de los colaboradores con respecto al proyecto colaborativo.


\section{Motivación personal}

Aunque la idea que llevó a la consecución de MiStuRe no es propia, sino que parte de una propuesta del tutor, personalmente no me resultó una propuesta carente de motivación, ni mucho menos extraña.


Durante el curso de las asignaturas de tercer y cuarto curso de Ingeniería denominadas IARO -Información Audiovisual en Redes de Ordenadores- y SAT -Servicios y Aplicaciones Telemáticas-, fuimos conejillos de indias, aunque no los primeros, en cursar dichas asignaturas siguiendo una metodología no muy común en el resto de las materias.


Dicha metodología dejaba en cierto modo de lado la típica lección magistral de pizarra, quedando reducida en muchas fases de estas asignaturas a un ajustado en el tiempo \textit{speech} sobre la materia tratada y posterior discusión de esta y de las dudas planteadas por los alumnos gracias al trabajo previo. El resto se basaba en el trabajo y la visualización de un pequeño dosier de materiales, referencias y ejemplos, a menudo interactivos, presentados telemáticamente en las jornadas previas a la clase correspondiente, y la realización de tests online sobre la propia materia a continuación de realizar nuestro trabajo con este dosier.


La parte que está relacionada con MiStuRe, esta radicada en la vertiente práctica de estas asignaturas. Esta consistía en la realización de pequeños ejercicios de programación, incrementales, y en la corrección semiautomatizada de estos, que a cambio nos aportaba cierta información adicional sobre nuestro desempeño programando, y que probablemente en más de un caso nos generó curiosidad para mejorar nuestra calidad de escritura o estilo de código, por ejemplo.


En resumen, que dada la experiencia propia en el lado del alumno, surgió el impulso vital para pasar a formar parte de ello y tratar de mejorar las herramientas para potenciar esta metodología de impartir clases teóricas y prácticas.


Otra motivación que me llevo a adoptar la realización de un proyecto así, fueron las múltiples posibilidades que pueden permitir la implementación de un escenario así en cuanto a temas y tecnologías con los que interactuar.


\section{Estructura de la memoria}

Esta memoria, relativa al Proyecto Fin de Carrera denominado como MiStuRe, al cual desde este momento denominaremos como Misture o proyecto para simplificar, consta de seis capítulos, apéndices sobre las nociones de uso de Misture y la instalación de las debidas dependencias para su correcto funcionamiento, y de un apartado de bibliografía.


En el primer capítulo, que estamos concluyendo con esta sección, se dedica a la introducción del proyecto Misture, la motivación del autor para llevarlo a cabo, y esta explicación de la estructura que se está realizando.


En un segundo capítulo, el de objetivos, se detalla el problema a resolver con el proyecto Misture, concretamente mejorar la funcionalidad de unos \textit{scripts} piloto, cuya estructura se analizará en general, y en base a este punto de partida, definiremos los objetivos.


En sucesivas secciones de este capítulo, también se detallan los objetivos perseguidos por el tutor tanto como por el autor con la realización de Misture, tanto a nivel práctico como a nivel teórico. Asimismo, se detallarán los requisitos funcionales que se propusieron inicialmente para Misture, que constituyen en sí mismos otro objetivo.


En el capítulo tercero, denominado ``Estado del Arte'', nos centraremos en detallar aquellas tecnologías, librerías o herramientas clave, así como unas reseñas sobre los ``Mining Software Repositories''.


En el capítulo cuarto, se describirá el diseño y arquitectura general de Misture, sus componentes, que elementos y utilidades nos sirven en su implementación y la estructura de las BBDD utilizadas.


El penúltimo capítulo, ``Resultados'', resume que objetivos se han intentado cumplir finalmente, en qué grado se han cumplido o en qué estado han quedado.


La cuenta de capítulos queda cerrada con aquel denominado ``Conclusiones'', donde recopilamos las ideas y resoluciones a las que hemos llegado con la elaboración de este proyecto.


Tras el desarrollo de la memoria, adjuntamos los apéndices con aclaraciones de uso y particularidades de las dependencias necesarias para la ejecución del proyecto, y la bibliografía con las principales referencias utilizadas a lo largo del desarrollo de Misture.


