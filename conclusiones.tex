\cleardoublepage
\chapter{Conclusiones}
\label{chap:conclusiones}

\section{Lecciones aprendidas}

Una de las principales lecciones aprendidas, consiste en entender la complejidad que resulta del diseño de herramientas que permitan la automatización de tareas.


La posibilidad de generar estos automatismos en casos de corrección lo más generales posibles debe estar muy bien estudiada en sus elementos o circunstancias para evitar encontrarse ante un caso complicado de manejar y de controlar.


Sino, o no es una tarea netamente automatizable, o debe ser considerada en tareas más reducidas y manejables y cuyos elementos característicos sean lo más homogéneos entre sí.


También puedo considerar que he aprendido a evaluar mejor que proyectos o APIs debo elegir, y he adquirido la noción de lo complejo que puede llegar a ser mantener en el tiempo pequeños proyectos de código abierto por parte de pequeños grupos de desarrolladores, a menudo de forma altruista en su tiempo libre.


El mayor ejemplo de esto que digo es el cambio que tuve que realizar del ODM \textit{mongoengine} por PyMODM, el primero bastante potente y con bastante historial detrás y que potencialmente podía facilitarme mucho una integración con Django.


Sin embargo, con el paso del tiempo y al utilizar más funcionalidades de este ODM y algunos fallos heredados de diseño de este, empezó a generar diversos problemas y comportamientos inesperados para los que no encontraba solución, teniendo que cambiar y llevándome a sufrir una pérdida de tiempo considerable.


No obstante, otra opción es elegir uno de estos proyectos que a uno le resulte atractivo y sus colaboradores estén abiertos a colaboraciones, y colaborar con ellos.


Otra importante lección aprendida por mi, es quizás reconocer mi exceso de perfeccionismo en la realización de este tipo de tareas, en ocasiones por encima de la capacidad técnica que pueda tener en el momento dado, o la disponibilidad de tiempo o recursos para llevar a cabo su ejecución.


En muchas ocasiones, no palpar adecuadamente estos límites junto al ansia perfeccionista, me ha llevado a callejones sin salida, códigos complejos que he tenido que desechar, y derivas de tiempo.


No obstante, por otra parte, esa característica propia de mí, aplicada al mundillo profesional, me ha llevado a ser muy bien valorado como analista en varios de los proyectos en los que he trabajado. Sin embargo, la cantidad de estrés que en ocasiones genera esto, es excesivo.

\newpage
\section{Conocimientos aplicados}

De una u otra forma, en la consecución de este proyecto, se han aplicado conocimientos adquiridos a lo largo de mi etapa universitaria cursando Ingeniería de Telecomunicación e Ingeniería Técnica en Informática de Sistemas.


Por una parte, hubo que recordar o aplicar conocimientos de la propia asignatura IARO, ya que alguna de las muestras de actividades empleadas para probar las funcionalidades correspondía a prácticas de contenido semejante a IARO.


Asimismo, nos aprovechamos del conocimiento adquirido en BBDD en las asignaturas de dicha temática. También de su uso en un \textit{framework} como Django, cuya destreza adquirimos en materias como Sistemas y Aplicaciones Telemáticas -SAT-.


También nos han sido útiles los conocimientos de otras materias como Ingeniería del Software y de Programación Orientada a Objetos.


No obstante, una vez afanados en encontrar relaciones, podríamos encontrarlas en todas las materias asociadas a la programación, BBDD, al desarrollo de software o a sistemas y aplicaciones web o de multimedia.

\newpage
\section{Conocimientos adquiridos}


Durante la investigación y desarrollo de Misture, he entrado en contacto con un buen número de conceptos, utilidades y tecnologías, algunas de las cuáles incluso excedían con creces la capacidad y tiempo que deben otorgarse a un Proyecto Fin de Carrera y no he llegado a aplicar en el proyecto y por tanto mencionar aquí.


Entre aquellos conocimientos adquiridos y practicados se encuentran:


\begin{itemize}
\item El DBMS MongoDB, además de los conceptos relativos a bases de datos NO-SQL o no relacionales.

\item Conocimientos sobre la herramienta Git y el manejo de repositorios distribuidos, además del uso de los servicios Github y Gitlab.

\item Aunque Python no era desconocido para mí, se puede decir que también he aprendido acerca de Python 3, sus diferencias con respecto la versión 2, y las características nuevas que trae.

\item También se han adquirido conocimientos del uso y funcionamiento de diferentes herramientas de análisis estático, dinámico y de estilo del código, especialmente de aquellas orientadas al lenguaje Python: Pep8, Pylint, Pymetrics, Flake8, Pyflakes, etc.

\end{itemize}

\newpage
\section{Mejoras o trabajos futuros}


En este apartado, se describen las posibles mejoras o líneas futuras que permitan un mayor desarrollo del proyecto obteniendo mejoras en el mismo.


Dado que personalmente observé una gran cantidad de opciones donde poder profundizar, pero que por complejidad o falta de tiempo no ha sido posible efectuar, voy a detallar algunas funcionalidades o mejoras posibles que se pueden aplicar al proyecto propuesto:


\begin{itemize}
\item Ampliar el uso de la interfaz web como cliente para poderse configurar desde esta interfaz la corrección de una actividad.

\item Ampliar las funcionalidades para hacer posible análisis entre de las diferentes actividades que con el tiempo se hayan corregido y pasen a formar parte de nuestro propio repositorio de correcciones.

\item Emplear las capacidades de módulos o librerías como Ast o Astor para posibilitar funcionalidades más potentes en los análisis de código fuente y su estructura, o para realizar pequeñas correcciones asistidas del código fuente sobre pequeños errores frecuentes de los alumnos que pueden dificultar la ejecución de tareas como las pruebas de ejecución.

\item Añadir nuevas formas de comunicación a la utilidad. Por ejemplo, añadir comunicación vía redes sociales tales como Twitter.

\item Integrar capacidades propias para detección de plagio.

\item Otra posibilidad a explorar es integrar la funcionalidad de MOSS en el propio Misture, debido a la existencia de \textit{scripts} cliente y de librerías para la visualización y análisis de los resultados del análisis.
\end{itemize}
