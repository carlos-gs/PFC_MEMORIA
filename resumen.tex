%%%% Resumen

\chapter*{Resumen}
%\addcontentsline{toc}{chapter}{Resumen} % si queremos que aparezca en el índice
\markboth{RESUMEN}{RESUMEN} % encabezado

En el contexto actual de las enseñanzas de Grado, el profesorado tiene una doble exigencia. Por un lado, proporcionar una docencia con un método de evaluación continua y muy dinámica. Por otro, la exigencia de una intensa actividad investigadora. 

En este escenario, una materia tiene como pretensión unos objetivos generales, unos objetivos específicos, y dotar de un conjunto de habilidades al alumno. Resulta positiva cualquier mejora o sistema que refuerce al profesor, optimice el tiempo disponible para evaluar, o incremente la frecuencia y cantidad de información que realimenta a alumno y profesor respecto al estado de estos objetivos.

Por otra parte, nos encontramos con disciplinas como la programación, donde es posible el tratamiento automático, la obtención de patrones o de mediciones de los entregables donde los alumnos demuestran su destreza y su evolución. En éstas, puede ser útil la aplicación de métodos o el uso de sistemas que permitan extraer sistemáticamente información valiosa con más rapidez que una metodología tradicional. La utilidad está en permitir de una forma más dinámica y objetiva la transmisión y el refuerzo de conocimientos, la detección de los errores y de los hábitos mejorables a nivel individual y grupal.

Tomando como referencia unas experiencias previas en unas asignaturas en las que participé, donde se emplearon estos conceptos, se va a intentar construir una propuesta de un sistema más automatizado y generalizable. Este sistema estará orientado a alumnos con un nivel básico de programación, en el que aun están formando sus destrezas y hábitos, y será aplicado a materias donde la programación es un soporte para el aprendizaje de otros conceptos como los relativos a redes y servicios web.

Eminentemente, estaremos en un ecosistema donde los alumnos manejarán el lenguaje Python, caracterizado por su simplicidad y alta interoperabilidad, y la herramienta colaborativa Git como elementos comunes.



%%%%%%%%%%%%%%%%%%%%%%%%%%%%%%%%%%%%%%%%%%%%%%%%%%%%%%%%%%%%%%%%%%%%%%%%%%%%%%%%