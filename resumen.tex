%%%% Resumen

\chapter*{Resumen}
%\addcontentsline{toc}{chapter}{Resumen} % si queremos que aparezca en el índice
\markboth{RESUMEN}{RESUMEN} % encabezado

En el contexto actual de las enseñanzas de Grado, con un profesorado con una doble exigencia, o con la autoexigencia de proporcionar una docencia con una metodología de evaluación continua y de forma muy dinámica a la vez del requerimiento de una intensa actividad investigadora. Se intuye como más que probable, que cualquier mejora o sistema que refuerce, optimice el tiempo disponible o incremente la frecuencia y la cantidad de información realimentada con respecto a los objetivos principales de una materia, tanto como de objetivos secundarios y habilidades asociadas a ellos entre los extremos típicos del acto educativo, sea positivo para ambos.\\


Por otra parte, en el contexto descrito, en disciplinas tales como la programación donde sea posible el tratamiento automático, obtención de patrones, o mediciones de los entregables donde los alumnos demuestran su destreza y evolución. Puede ser útil la aplicación de métodos o el uso de sistemas que permitan sistematizar y extraer información valiosa con más rapidez que con la metodología tradicional. Y que permita de forma más dinámica y objetiva la transmisión y refuerzo de los conocimientos y la detección de errores o hábitos a mejorar tanto a nivel individual como a nivel grupal.\\


En este escenario, y teniendo como referencia unas experiencias previas en unas asignaturas cursadas donde se emplearon estos conceptos, se va a intentar construir una propuesta de un sistema más automatizado y generalizable, orientado a alumnos con un nivel donde todavía están por formar parte de las destrezas y hábitos de programación y materias donde se aplique la programación a pequeñas aplicaciones  orientadas al aprendizaje de otros conceptos como redes y servicios web.\\


Eminentemente, trataremos con un ecosistema donde los alumnos manejarán el lenguaje Python, caracterizado por su simplicidad y alta interoperabilidad, y la herramienta colaborativa Git como elementos comunes.



%%%%%%%%%%%%%%%%%%%%%%%%%%%%%%%%%%%%%%%%%%%%%%%%%%%%%%%%%%%%%%%%%%%%%%%%%%%%%%%%